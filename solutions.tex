\documentclass[letter]{article}
\usepackage[utf8]{inputenc}
\usepackage{amsmath}
\usepackage{amsthm}
\usepackage{amssymb}
\usepackage{mathrsfs}
%\usepackage{algpseudocode}
\usepackage{mathtools}
\usepackage{stmaryrd}
\usepackage{fancyhdr}
\usepackage{siunitx}
\usepackage[bb=boondox]{mathalfa}
\usepackage{tkz-berge}
\usepackage{enumerate}
\usepackage{stackengine}
\usepackage{algorithm}
\usepackage{algorithmic}
\usepackage{comment}
\usepackage{bussproofs}
% DEFINE DEVIDES
\makeatletter
\def\localbig#1#2{%
	\sbox\z@{$\m@th#1
		\sbox\tw@{$#1()$}%
		\dimen@=\ht\tw@\advance\dimen@\dp\tw@
		\nulldelimiterspace\z@\left#2\vcenter to1.2\dimen@{}\right.
		$}\box\z@}

\newcommand{\divides}{\mathrel{\mathpalette\dividesaux\relax}}
\newcommand{\ndivides}{\mathrel{\mathpalette\ndividesaux\relax}}

\newcommand{\dividesaux}[2]{\mbox{$\m@th#1\localbig{#1}|$}}
\newcommand{\ndividesaux}[2]{%
	\mkern.5mu
	\ooalign{%
		\hidewidth$\m@th#1\localbig{#1}|$\hidewidth\cr
		$\m@th#1\nmid$\cr%
	}%
}
\makeatother
% DONE DIVIDES

\newcommand{\abs}[1]{\left\lvert#1\right\rvert}

%opening
\title{Logic Reading Group\\Solutions to \textit{Logic and Structure}}
\date{Summer 2020}

\newcommand{\vsep}{\textpipe\ }


\newcommand{\Span}{\text{span}}
\newcommand{\diag}{\text{diag}}
\newtheorem{problem}{Problem}
\newtheorem{theorem}{Theorem}
\newtheorem{lemma}{Lemma}
\theoremstyle{definition}
\newtheorem{definition}{Definition}
\newenvironment{solution}
{\begin{proof}[Solution]}
	{\end{proof}}

\newcommand*{\lsim}{\mathord{\sim}}

\newcommand*{\braces}[1]{\left\{#1\right\}}
\newcommand*{\bracks}[1]{\left\lbrack{#1}\right\rbrack}
\newcommand*{\parens}[1]{\left({#1}\right)}

\newcommand\overla[3][0pt]{\stackengine{0pt}{#3}{\kern#1#2}{O}{c}{F}{F}{L}}
\newcommand\overlam[3][0pt]{\text{\stackengine{0pt}{$#3$}{\kern#1$#2$}{O}{c}{F}{F}{L}}}

\def\shrinkage{2.1mu}
\def\vecsign{\mathchar"017E}
\def\dvecsign{\smash{\stackon[-1.95pt]{\mkern-\shrinkage\vecsign}{\rotatebox{180}{$\mkern-\shrinkage\vecsign$}}}}
\def\dvec#1{\def\useanchorwidth{T}\stackon[-4.2pt]{#1}{\,\dvecsign}}
\usepackage{stackengine}
\stackMath
\usepackage{graphicx}


\DeclareMathOperator*{\bigdoublewedge}{\bigwedge\mkern-15mu\bigwedge}
\DeclareMathOperator*{\bigdoublevee}{\bigvee\mkern-15mu\bigvee}


\begin{document}

\maketitle
 
\newpage
\begin{problem}[1.1.2]
    Show that $(( \to \not \in \textrm{PROP}$.
\end{problem}
\begin{solution}
\end{solution}

\begin{problem}[1.1.9] Show that a proposition with $n$ connectives has at most $2n + 1$ formulas.
\end{problem}
\begin{solution}
\end{solution}

\begin{problem}[1.1.10] Show that for $\textrm{PROP}$ we have a unique decomposition theorem: for each non-atomic proposition $\sigma$ either there are two propositions $\phi$ and $\psi$ such that $\sigma = \phi \square \psi$ or there is a proposition $\phi$ such that $\sigma = \neg \phi$.
\end{problem}
\begin{solution}
\end{solution}

\begin{problem}[1.2.1] Check by the truth table method which of the following propositions are tautologies:
    \begin{enumerate}[(a)]
        \item $(\neg \phi \lor \psi) \iff (\psi \to \phi)$
        \item $\phi \to ((\psi \to \sigma)
                          \to
                          ((\phi \to \psi)
                            \to (\phi \to \sigma)
                            ))$
        \item $(\phi \to \neg \phi) \iff \neg \phi$
        \item $\neg (\phi \to \neg \phi)$
    \end{enumerate}
\end{problem}
\begin{solution}
\end{solution}

\begin{problem}[1.2.2] Show
    \begin{enumerate}[(a)]
        \item $\phi \models \phi$
        \item $\phi \models \psi \textrm{ and } \psi \models \sigma \> \implies \> \phi \models \sigma$
        \item $\models \phi \to \psi \> \iff \> \phi \models \psi$
    \end{enumerate}
\end{problem}
\begin{solution}
\end{solution}

 
\newcommand{\den}[1]{\llbracket #1 \rrbracket_v}
\begin{problem}[1.2.5]
    Show
 \begin{enumerate}[(a)]
     \item $\den{\phi \land \psi} = \den{\phi} \cdot \den{\psi}$
     \item $\den{\phi \lor \psi} = \den{\phi} + \den{\psi} - \den{\phi} \cdot \den\psi$
     \item $\den{\phi \to \psi} = 1 - \den{\phi} + \den{\phi} \cdot \den{\psi}$
     \item $\den{\phi \leftrightarrow \psi} = 1 - \big|
     \den{\phi} - \den{\psi} \big|$
 \end{enumerate}
\end{problem}
\begin{solution}
    
\end{solution}

\begin{problem}[1.3.1] Show by `algebraic' means
 \begin{enumerate}[(a)]
     \item $\models (\phi \to \psi) \leftrightarrow (\neg \psi \to \neg \phi)$
     \item $\models (\phi \to \psi) \land (\psi \to \sigma) \to (\phi \to \sigma)$
     \item $\models (\phi \to (\psi \land \neg \psi) \to \neg \phi$
     \item $\models (\phi \to \neg \psi) \to \neg \phi$
     \item $\models \neg (\phi \land \neg \phi)$
     \item $\models \phi \to (\psi \to \phi \land \psi)$
     \item $\models ((\phi \to \psi) \to \phi) \to \phi$
 \end{enumerate}
\end{problem}
\begin{solution}
\end{solution}

\begin{problem}[1.3.3] Show that $\{ \neg \}$ is not a functionally complete set of connectives. Same for $\{\to, \lor\}$ (hint: show that each formula $\phi$ with only $\to$ and $\lor$ there is a valuation v such that $\llbracket \phi \rrbracket_v = 1$)
\end{problem}
\begin{solution}
\end{solution}

\begin{problem}[1.3.4] Show that the Sheffer stroke, $\uparrow$, forms a functionally complete set. (hint: $\models \neg \phi \leftrightarrow \phi \uparrow \phi$).
\end{problem}
\begin{solution}
\end{solution}


\begin{problem}[1.3.10] Determine conjunctive and disjunctive normal forms for
\begin{enumerate}[(a)]
    \item $\neg (\phi \leftrightarrow \psi)$
    \item $((\phi \to \psi) \to \psi) \to \psi$
    \item $(\phi \to (\phi \land \neg \psi)) \land (\psi \to (\psi \land \neg \phi))$
\end{enumerate}
\end{problem}
\begin{solution}
\end{solution}

\begin{problem}[1.3.11] Give a criterion for a conjunctive normal form to be a tautology.
\end{problem}
\begin{solution}
\end{solution}

\begin{problem}[1.4.1] Show that the following propositions are derivable (note: see at the bottom for guidance on formatting proof trees)
\begin{enumerate}[(a)]
    \item $\varphi \to \varphi$
    \item $\bot \to \varphi$
    \item $\neg (\varphi \land \neg \varphi)$
    \item $(\varphi \to \psi) \leftrightarrow \neg (\varphi \land \neg \psi)$
    \item $(\varphi \land \psi) \leftrightarrow \neg (\varphi \to \neg \psi)$
    \item $\varphi \to (\psi \to (\varphi \land \psi))$
\end{enumerate}
\end{problem}
\begin{solution}\strut
\begin{enumerate}[(a)]

    \item\strut
    \begin{prooftree}
        \AxiomC{$\bracks{\varphi}^1$}

        \RightLabel{$\to I_1$}
        \UnaryInfC{$\varphi \to \varphi$}
    \end{prooftree}

    \item\strut
    \begin{prooftree}
        \AxiomC{$\bracks{\bot}^1$}

        \RightLabel{$\bot$}
        \UnaryInfC{$\varphi$}

        \RightLabel{$\to I_1$}
        \UnaryInfC{$\bot \to \varphi$}
    \end{prooftree}
    
    \item Here is (c) for notational reference:
    \begin{prooftree}
        \AxiomC{$[ \varphi \land \neg \varphi]^1$}
        \RightLabel{$\land E (l)$}
        \UnaryInfC{$\varphi$}
        \AxiomC{$[ \varphi \land \neg \varphi]^1$}
        \RightLabel{$\land E (r)$}
        \UnaryInfC{$\neg \varphi$}
        \RightLabel{$\to E$}
        \BinaryInfC{$\bot$}
        \RightLabel{$\to I_1$}
        \UnaryInfC{$\neg (\varphi \land \neg \varphi)$}
    \end{prooftree}

    \item Let
    {
        \AxiomC{$\mathcal D_1$}
        \noLine
        \UnaryInfC{$\parens{\varphi \to \psi} \to \lnot\parens{\varphi \land \lnot \psi}$}
        \DisplayProof
    }
    $=$
    \begin{prooftree}
                \AxiomC{$\bracks{\varphi \land \lnot \psi}^1$}
                \RightLabel{$\land E (l)$}
                \UnaryInfC{$\varphi$}

                \AxiomC{$\bracks{\varphi \to \psi}^2$}

            \RightLabel{$\to E$}
            \BinaryInfC{$\psi$}

            \AxiomC{$\bracks{\varphi \land \lnot \psi}^1$}
            \RightLabel{$\land E (r)$}
            \UnaryInfC{$\lnot \psi$}

        \RightLabel{$\to E$}
        \BinaryInfC{$\bot$}

        \RightLabel{$\to I_1$}
        \UnaryInfC{$\lnot\parens{\varphi \land \lnot \psi}$}

        \RightLabel{$\to I_2$}
        \UnaryInfC{$\parens{\varphi \to \psi} \to \lnot\parens{\varphi \land \lnot \psi}$}
    \end{prooftree}
    and let
    {
        \AxiomC{$\mathcal D_2$}
        \noLine
        \UnaryInfC{$\lnot \parens{\varphi \land \lnot \psi} \to \parens{\varphi \to \psi}$}
        \DisplayProof
    }
    $=$
    \begin{prooftree}
                \AxiomC{$\bracks{\varphi}^4$}
                \AxiomC{$\bracks{\lnot \psi}^3$}

            \RightLabel{$\land I$}
            \BinaryInfC{$\phi \land \lnot \psi$}

            \AxiomC{$\bracks{\lnot\parens{\varphi \land \lnot \psi}}^5$}

        \RightLabel{$\to E$}
        \BinaryInfC{$\bot$}

        \RightLabel{RAA${}_3$}
        \UnaryInfC{$\psi$}

        \RightLabel{$\to I_4$}
        \UnaryInfC{$\varphi \to \psi$}

        \RightLabel{$\to I_5$}
        \UnaryInfC{$\lnot \parens{\varphi \land \lnot \psi} \to \parens{\varphi \to \psi}$}
    \end{prooftree}
    Then
    \begin{prooftree}
            \AxiomC{$\mathcal D_1$}
            \noLine
            \UnaryInfC{$\parens{\varphi \to \psi} \to \lnot\parens{\varphi \land \lnot \psi}$}

            \AxiomC{$\mathcal D_2$}
            \noLine
            \UnaryInfC{$\lnot \parens{\varphi \land \lnot \psi} \to \parens{\varphi \to \psi}$}

        \RightLabel{$\land I$}
        \BinaryInfC{$\parens{\varphi \to \psi} \leftrightarrow \lnot\parens{\varphi \land \lnot \psi}$}
    \end{prooftree}

    \item Let
    {
        \AxiomC{$\mathcal D_1$}
        \noLine
        \UnaryInfC{$\parens{\varphi \land \psi} \to \lnot\parens{\varphi \to \lnot \psi}$}
        \DisplayProof
    }
    $=$
    \begin{prooftree}
            \AxiomC{$\bracks{\varphi \land \psi}^2$}
            \RightLabel{$\land E (r)$}
            \UnaryInfC{$\psi$}

                \AxiomC{$\bracks{\varphi \land \psi}^2$}
                \RightLabel{$\land E (l)$}
                \UnaryInfC{$\varphi$}

                \AxiomC{$\bracks{\varphi \to \lnot \psi}^1$}

            \RightLabel{$\to E$}
            \BinaryInfC{$\lnot \psi$}

        \RightLabel{$\to E$}
        \BinaryInfC{$\bot$}

        \RightLabel{$\to I_1$}
        \UnaryInfC{$\lnot\parens{\varphi \to \lnot \psi}$}

        \RightLabel{$\to I_2$}
        \UnaryInfC{$\parens{\varphi \land \psi} \to \lnot\parens{\varphi \to \lnot \psi}$}
    \end{prooftree}
    let
    {
        \AxiomC{$\lnot\parens{\varphi \to \lnot \psi}$}
        \noLine
        \UnaryInfC{$\mathcal D_2$}
        \noLine
        \UnaryInfC{$\varphi$}
        \DisplayProof
    }
    $=$
    \begin{prooftree}
                \AxiomC{$\bracks{\varphi}^3$}
                \AxiomC{$\bracks{\lnot \varphi}^4$}

            \RightLabel{$\to E$}
            \BinaryInfC{$\bot$}
            \RightLabel{$\bot$}
            \UnaryInfC{$\lnot \psi$}
            \RightLabel{$\to I_3$}
            \UnaryInfC{$\varphi \to \lnot \psi$}

            \AxiomC{$\lnot\parens{\varphi \to \lnot \psi}$}

        \RightLabel{$\to E$}
        \BinaryInfC{$\bot$}

        \RightLabel{RAA${}_4$}
        \UnaryInfC{$\varphi$}
    \end{prooftree}
    and let
    {
        \AxiomC{$\lnot\parens{\varphi \to \lnot \psi}$}
        \noLine
        \UnaryInfC{$\mathcal D_3$}
        \noLine
        \UnaryInfC{$\psi$}
        \DisplayProof
    }
    $=$
    \begin{prooftree}
            \AxiomC{$\bracks{\lnot \psi}^5$}
            \RightLabel{$\to I$}
            \UnaryInfC{$\varphi \to \lnot \psi$}

            \AxiomC{$\lnot\parens{\varphi \to \lnot \psi}$}

        \RightLabel{$\to E$}
        \BinaryInfC{$\bot$}

        \RightLabel{RAA${}_5$}
        \UnaryInfC{$\psi$}
    \end{prooftree}
    Then
    \begin{prooftree}
            \AxiomC{$\mathcal D_1$}
            \noLine
            \UnaryInfC{$\parens{\varphi \land \psi} \to \lnot\parens{\varphi \to \lnot \psi}$}

                \AxiomC{$\bracks{\lnot\parens{\varphi \to \lnot \psi}}^6$}
                \noLine
                \UnaryInfC{$\mathcal D_2$}
                \noLine
                \UnaryInfC{$\varphi$}

                \AxiomC{$\bracks{\lnot\parens{\varphi \to \lnot \psi}}^6$}
                \noLine
                \UnaryInfC{$\mathcal D_3$}
                \noLine
                \UnaryInfC{$\psi$}

            \RightLabel{$\land I$}
            \BinaryInfC{$\varphi \land \psi$}
            \RightLabel{$\to I_6$}
            \UnaryInfC{$\lnot \parens{\varphi \to \lnot \psi} \to \parens{\varphi \land \psi}$}

        \RightLabel{$\land I$}
        \BinaryInfC{$\parens{\varphi \land \psi} \leftrightarrow \lnot\parens{\varphi \to \lnot \psi}$}
    \end{prooftree}

    \item\strut
    \begin{prooftree}
            \AxiomC{$\bracks{\varphi}^2$}

            \AxiomC{$\bracks{\psi}^1$}

        \RightLabel{$\land I$}
        \BinaryInfC{$\varphi \land \psi$}

        \RightLabel{$\to I_1$}
        \UnaryInfC{$\psi \to \parens{\varphi \land \psi}$}

        \RightLabel{$\to I_2$}
        \UnaryInfC{$\varphi \to \parens{\psi \to \parens{\varphi \land \psi}}$}
    \end{prooftree}
\end{enumerate}
\end{solution}

\begin{problem}[1.4.3] Show
    \begin{enumerate}[(a)]
        \item $\varphi \vdash \neg (\neg \varphi \land \psi)$
        \item $\neg (\varphi \land \neg \psi), \varphi \vdash \psi$
        \item $\neg \varphi \vdash (\varphi \to \psi) \leftrightarrow \neg \varphi$
        \item $\vdash \varphi \implies \vdash \psi \to \varphi$
        \item $\neg \varphi \vdash \varphi \to \psi$
    \end{enumerate}
\end{problem}
\begin{solution}\strut
\begin{enumerate}[(a)]
    \item\strut
    \begin{prooftree}
            \AxiomC{$\varphi$}

                \AxiomC{$\bracks{\lnot \varphi \land \psi}^1$}
                \RightLabel{$\land E (l)$}

            \UnaryInfC{$\lnot \varphi$}

        \RightLabel{$\to E$}
        \BinaryInfC{$\bot$}

        \RightLabel{$\to I_1$}
        \UnaryInfC{$\lnot \parens{\lnot \varphi \land \psi}$}
    \end{prooftree}

    \item\strut
    \begin{prooftree}
                \AxiomC{$\varphi$}
                \AxiomC{$\bracks{\lnot \psi}^1$}

            \RightLabel{$\land I$}
            \BinaryInfC{$\varphi \land \lnot \psi$}

            \AxiomC{$\lnot \parens{\varphi \land \lnot \psi}$}

        \RightLabel{$\to E$}
        \BinaryInfC{$\bot$}

        \RightLabel{RAA${}_1$}
        \UnaryInfC{$\psi$}
    \end{prooftree}

    \item\strut
    \begin{prooftree}
            \AxiomC{$\lnot \varphi$}
            \RightLabel{$\to I$}
            \UnaryInfC{$\parens{\varphi \to \psi} \to \lnot \varphi$}

                \AxiomC{$\bracks{\varphi}^1$}
                \AxiomC{$\bracks{\lnot \varphi}^2$}

            \RightLabel{$\to E$}
            \BinaryInfC{$\bot$}
            \RightLabel{$\bot$}
            \UnaryInfC{$\psi$}
            \RightLabel{$\to I_1$}
            \UnaryInfC{$\varphi \to \psi$}
            \RightLabel{$\to I_2$}
            \UnaryInfC{$\lnot\varphi \to \parens{\varphi \to \psi}$}

        \RightLabel{$\land I$}
        \BinaryInfC{$\parens{\varphi \to \psi} \leftrightarrow \lnot \varphi$}
    \end{prooftree}

    \item
    Assume there is a derivation
    {\AxiomC{$\mathcal D$} \noLine \UnaryInfC{$\varphi$} \DisplayProof}.
    \begin{prooftree}
        \AxiomC{$\mathcal D$}

        \noLine
        \UnaryInfC{$\varphi$}

        \RightLabel{$\to I$}
        \UnaryInfC{$\psi \to \varphi$}
    \end{prooftree}

    \item\strut
    \begin{prooftree}
        \AxiomC{$\bracks{\varphi}^1$}
        \AxiomC{$\lnot \varphi$}

        \RightLabel{$\to E$}
        \BinaryInfC{$\bot$}

        \RightLabel{$\bot$}
        \UnaryInfC{$\psi$}

        \RightLabel{$\to I_1$}
        \UnaryInfC{$\varphi \to \psi$}
    \end{prooftree}
\end{enumerate}
\end{solution}

\begin{problem}[1.4.4] Show
    \begin{enumerate}[(a)]
        \item $[(\phi \to \psi) \to (\phi \to \sigma)] \to [ \phi \to \psi \to \sigma]$
        \item $((\phi \to \psi) \to \phi) \to \phi$
    \end{enumerate}
\end{problem}
\begin{solution}
\end{solution}

\begin{problem}[1.4.10] Give an inductive definition of the relation $\vdash$ (c.f. Lemma 1.4.3). Show this relation coincides with the derived relation of Definition 1.4.2. Conclude that if $\Gamma \vdash \phi$, then $\Gamma$ contains a finite $\Delta$ such that $\Delta \vdash \phi$.
\end{problem}
\begin{solution}
\end{solution}


\begin{problem}[1.5.1] Check which of the following sets are consistent
    \begin{enumerate}[(a)]
        \item $\{\neg p_1 \land p_2 \to p_0, p_1 \to (\neg p_1 \to p_2), p_0 \leftrightarrow \neg p_2\}$
        \item $\{p_0 \to p_1. p_1 \to p_2, p2 \to p_3, p_3 \to \neg p_0\}$
        \item $\{p_0 \to p_1, p_0 \land p_2 \to p_1 \land p_3, p_0 \land p_2 \land p_4 \to p_1 \land p_3 \land p_5 \}$
    \end{enumerate}
\end{problem}
\begin{solution}
\end{solution}

\begin{problem}[1.5.2] Show that the following are equivalent:
    \begin{enumerate}[(a)]
        \item $\{\phi_1 \ldots \phi_n\}$ is consistent
        \item $\not \vdash \neg (\phi_1 \land \ldots \land \phi_n)$
        \item $\not \vdash \phi_1 \land \phi_2 \ldots \land \phi_{n-1}\to \neg \phi_n$
    \end{enumerate}
\end{problem}
\begin{solution}
\end{solution}

\begin{problem}[1.5.3] $\phi$ is \textit{independent} from $\Gamma$ if
    $\Gamma \not \vdash \phi$ and $\Gamma \not \vdash \neg \phi$. Show that $p_1 \to p_2$ is independent from $\{p_1 \leftrightarrow p_0 \land \neg p_2, p_2 \to p_0\}$
\end{problem}
\begin{solution}
\end{solution}

\begin{problem}[1.5.6] Show that a consistent set $\Gamma$ is maximally consistent if either $\phi \in \Gamma$ or $\neg \phi \in \Gamma$ for all $\phi$.
\end{problem}
\begin{solution}
\end{solution}


\begin{problem}[1.5.10] Show $\textrm{Cons}(\Gamma) = \{ \sigma | \Gamma \vdash \sigma \}$ is maximally consistent if and only if $\Gamma$ is complete.
\end{problem}
\begin{solution}
\end{solution}


\section{Notation Reference}

\begin{table}[H]
    \begin{tabular}{ll}
        \textbf{Notation} & \textbf{Usual meaning} \\
        $\to$ $\leftarrow$ $\lor$ $\land$ $\leftrightarrow$ $\neg$ & Object-level connectives \\
        $\uparrow$ & Scheffer stroke (nand)\\
        $\top$, $\bot$ & Object-level true, false \\
        $\implies$ $\impliedby$ $\iff$ $\neg$ & Meta-level connectives \\
        
        $\llbracket \phi \rrbracket_v$ & denotation of $\phi$ under valuation $v$\\
        $\phi \vdash \psi$ & $\phi$ proves $\psi$ \\
        $\phi \models \psi$ & $\phi$ models $\psi$ \\
        $\ulcorner \phi \urcorner$ & Quine quotations/G\"odel numbering \\
        $\bigwedge, \bigvee$ & \\        
        $\bigdoublevee$, $\bigdoublewedge$ & See Section 1.3 \\
        $\cup$, $\cap$, $\subset$, $\subseteq$, $\subsetneq$ & Set operations \\
        $\sqcup$, $\sqcap$, $\sqsubset$, $\sqsubseteq$ & Misc \\
\end{tabular}
\end{table}

\subsection{Typesetting proof trees}

\begin{prooftree}
    \AxiomC{}
    \RightLabel{identity}
    \UnaryInfC{$X \to X$}
\end{prooftree} 

\begin{prooftree}
\AxiomC{$H$}
\RightLabel{\scriptsize{$\lor$ introduction left}}
\UnaryInfC{$H \lor G$}
\end{prooftree} 

\begin{prooftree}
    \AxiomC{$H$}
    \AxiomC{$G$}
    \RightLabel{\scriptsize{$\land$ introduction}}
    \BinaryInfC{$H \land G$}
\end{prooftree} 


\begin{prooftree}
    \AxiomC{$A$}
    \AxiomC{$B$}
    \AxiomC{$C$}
    \RightLabel{\scriptsize{weird rule}}
    \TrinaryInfC{$D$}
\end{prooftree} 

\end{document}



