\documentclass[letter]{article}
\usepackage[utf8]{inputenc}
\usepackage{amsmath}
\usepackage{amsthm}
\usepackage{amssymb}
\usepackage{mathrsfs}
%\usepackage{algpseudocode}
\usepackage{mathtools}
\usepackage{stmaryrd}
\usepackage{fancyhdr}
\usepackage{siunitx}
\usepackage[bb=boondox]{mathalfa}
%\usepackage{tkz-berge} % Just uncomment if you need this
\usepackage{enumerate}
\usepackage{stackengine}
\usepackage{algorithm}
\usepackage{algorithmic}
\usepackage{comment}
\usepackage{bussproofs}
\usepackage{xcolor}
\usepackage{bm}
% DEFINE DEVIDES
\makeatletter
\def\localbig#1#2{%
	\sbox\z@{$\m@th#1
		\sbox\tw@{$#1()$}%
		\dimen@=\ht\tw@\advance\dimen@\dp\tw@
		\nulldelimiterspace\z@\left#2\vcenter to1.2\dimen@{}\right.
		$}\box\z@}

\newcommand{\divides}{\mathrel{\mathpalette\dividesaux\relax}}
\newcommand{\ndivides}{\mathrel{\mathpalette\ndividesaux\relax}}

\newcommand{\dividesaux}[2]{\mbox{$\m@th#1\localbig{#1}|$}}
\newcommand{\ndividesaux}[2]{%
	\mkern.5mu
	\ooalign{%
		\hidewidth$\m@th#1\localbig{#1}|$\hidewidth\cr
		$\m@th#1\nmid$\cr%
	}%
}
\makeatother
% DONE DIVIDES

\newcommand{\abs}[1]{\left\lvert#1\right\rvert}

%opening
\title{Logic Reading Group\\Solutions to \textit{Logic and Structure}}
\date{Summer 2020}

\newcommand{\vsep}{\textpipe\ }


\newcommand{\Span}{\text{span}}
\newcommand{\diag}{\text{diag}}
\newtheorem{problem}{Problem}
\newtheorem{theorem}{Theorem}
\newtheorem{lemma}{Lemma}
\theoremstyle{definition}
\newtheorem{definition}{Definition}
\newenvironment{solution}
{\begin{proof}[Solution]}
	{\end{proof}}

\renewcommand{\phi}{\varphi}

\newcommand*{\lsim}{\mathord{\sim}}

\newcommand*{\braces}[1]{\left\{#1\right\}}
\newcommand*{\bracks}[1]{\left\lbrack{#1}\right\rbrack}
\newcommand*{\parens}[1]{\left({#1}\right)}

\newcommand\overla[3][0pt]{\stackengine{0pt}{#3}{\kern#1#2}{O}{c}{F}{F}{L}}
\newcommand\overlam[3][0pt]{\text{\stackengine{0pt}{$#3$}{\kern#1$#2$}{O}{c}{F}{F}{L}}}

\def\shrinkage{2.1mu}
\def\vecsign{\mathchar"017E}
\def\dvecsign{\smash{\stackon[-1.95pt]{\mkern-\shrinkage\vecsign}{\rotatebox{180}{$\mkern-\shrinkage\vecsign$}}}}
\def\dvec#1{\def\useanchorwidth{T}\stackon[-4.2pt]{#1}{\,\dvecsign}}
\usepackage{stackengine}
\stackMath
\usepackage{graphicx}


\DeclareMathOperator*{\bigdoublewedge}{\bigwedge\mkern-15mu\bigwedge}
\DeclareMathOperator*{\bigdoublevee}{\bigvee\mkern-15mu\bigvee}


\begin{document}

\maketitle

\newpage
\begin{problem}[1.1.2]
    Show that $(( \to \not \in \textrm{PROP}$.
\end{problem}
\begin{solution}
\end{solution}

\begin{problem}[1.1.9] Show that a proposition with $n$ connectives has at most $2n + 1$ formulas.
\end{problem}
\begin{solution}
\end{solution}

\begin{problem}[1.1.10] Show that for $\textrm{PROP}$ we have a unique decomposition theorem: for each non-atomic proposition $\sigma$ either there are two propositions $\phi$ and $\psi$ such that $\sigma = \phi \square \psi$ or there is a proposition $\phi$ such that $\sigma = \neg \phi$.
\end{problem}
\begin{solution}
\end{solution}

\begin{problem}[1.2.1] Check by the truth table method which of the following propositions are tautologies:
    \begin{enumerate}[(a)]
        \item $(\neg \phi \lor \psi) \leftrightarrow (\psi \to \phi)$
        \item $\phi \to ((\psi \to \sigma)
                          \to
                          ((\phi \to \psi)
                            \to (\phi \to \sigma)
                            ))$
        \item $(\phi \to \neg \phi) \leftrightarrow \neg \phi$
        \item $\neg (\phi \to \neg \phi)$
    \end{enumerate}
\end{problem}
\begin{solution}
\end{solution}
\newcommand{\den}[1]{\llbracket #1 \rrbracket_v}
\begin{problem}[1.2.2] Show
    \begin{enumerate}[(a)]
        \item $\phi \models \phi$
        \item $\phi \models \psi \textrm{ and } \psi \models \sigma \> \implies \> \phi \models \sigma$
        \item $\models \phi \to \psi \> \iff \> \phi \models \psi$
    \end{enumerate}
\end{problem}
\begin{solution}
  \begin{enumerate}[(a)]
    \item By Definition 1.2.4 (iii), $\phi \models \phi$ iff $\forall v, \den {\phi} = 1 
    \Rightarrow \den{\phi} = 1$, which holds.
    \item Given that $\forall v, \den{\psi} = 1 \Rightarrow \den{\phi} = 1$ and
    $\forall v, \den{\phi} = 1 \Rightarrow \den{\psi} = 1$, if we have that $\forall v, \den{\phi} = 1$,
    by our first assumption, we have that $\den{\phi} = 1$. If we apply this conclusion to our second
    assumption, we get that $\den{\sigma} = 1$.
    \item First, let's show that $\models \phi \to \psi \Rightarrow \phi \models \psi$.
    Given that $\forall v, \den{\phi} \leq \den{\psi}$, forall $v$ where $\den{\phi} = 1$, 
    $1 \leq \den{\phi}$, so $\den{\phi} = 1$, since the possible values of valuations is ${0,1}$.
    Second, we show the other direction. From $\forall v, \den{\psi} = 1 \Rightarrow \den{\psi} = 1$,
    for all possible valuations of $\psi$ and $\phi$ where this statement holds true (from a quick analysis
    on the truth table for implication), $\den{\phi} \leq \den{\psi}$.
  \end{enumerate}
    
\end{solution}



\begin{problem}[1.2.5]
    Show
 \begin{enumerate}[(a)]
 \item $\den{\phi \land \psi} = \den{\phi} \cdot \den{\psi}$
 \item $\den{\phi \lor \psi} = \den{\phi} + \den{\psi} - \den{\phi} \cdot \den\psi$
 \item $\den{\phi \to \psi} = 1 - \den{\phi} + \den{\phi} \cdot \den{\psi}$
 \item $\den{\phi \leftrightarrow \psi} = 1 - \big|\den{\phi} - \den{\psi} \big|$
 \end{enumerate}
\end{problem}
\begin{solution}
  Since for any $\phi, \den{\phi} \in \{0, 1\}$, each of the problems follow by case analysis on $\den{\phi}$ and $\den{\psi}$.
  \begin{enumerate}[(a)]
  \item
    $\den{\phi \land \psi} = \min(\den{\phi}, \den{\psi}) = \den{\phi} \cdot \den{\psi}$
  \item
    $\den{\phi \lor \psi} = \max(\den{\phi}, \den{\psi}) = \den{\phi} + \den{\psi} - \den{\phi} \cdot \den\psi$
  \item
    $\den{\phi \to \psi} =
    \begin{cases}
      0 \text{ if $\den{\phi} = 1$ and $\den{\psi} = 0$} \\
      1 \text{ otherwise}
    \end{cases}
    = 1 - \den{\phi} + \den{\phi} \cdot \den{\psi}$
  \item
    $\den{\phi \leftrightarrow \psi} =
    \begin{cases}
      1 \text{ if $\den{\phi} = \den{\psi}$} \\
      0 \text{ otherwise}
    \end{cases}
    = 1 - \big|\den{\phi} - \den{\psi} \big|$
  \end{enumerate}
\end{solution}

\begin{problem}[1.3.1] Show by `algebraic' means
 \begin{enumerate}[(a)]
     \item $\models (\phi \to \psi) \leftrightarrow (\neg \psi \to \neg \phi)$
     \item $\models (\phi \to \psi) \land (\psi \to \sigma) \to (\phi \to \sigma)$
     \item $\models (\phi \to (\psi \land \neg \psi) \to \neg \phi$
     \item $\models (\phi \to \neg \psi) \to \neg \phi$
     \item $\models \neg (\phi \land \neg \phi)$
     \item $\models \phi \to (\psi \to \phi \land \psi)$
     \item $\models ((\phi \to \psi) \to \phi) \to \phi$
 \end{enumerate}
\end{problem}
\begin{solution}
\newcommand*\tolor{{$\to$/$\lor$}}
\newcommand*\colop[2]{\mathbin{\color{#1}#2}} % Color a binary op in red.
%
Let us start with some general remarks about the provenance of results used in
this exercise, so we don't have to spell it out every time.

A very common theorem to remove implication is
$(\phi \to \psi) \approx (\neg\phi \lor \psi)$
(Theorem 1.3.4 (b)), herafter referred to as the \tolor{} equivalence.
This is necessary because we don't have simple rules to directly reason about
implication algebraically.

Common forms of algebraic properties (\textit{e.g.}, commutativity,
distributivity, involutivity of $\neg$, De Morgan laws) are provided by
Theorem 1.3.1. We make generous use of associativity without mentioning it.

The law of excluded middle, $\models \phi \lor \neg \phi$,
is proved in Exercise 1.2.1 (f).

Colors will be used to highlight algebraic transformations in big expressions.

\begin{enumerate}[(a)]
  \item $\models (\phi \to \psi) \leftrightarrow (\neg \psi \to \neg \phi)$
%%% (a)
\begin{align*}
  & \phi \to \psi && \\
\approx{} & \neg \phi \lor \psi     && \text{by \tolor{}} \\
\approx{} & \psi \lor \neg \phi     && \text{by commutativity} \\
\approx{} & \neg \neg \psi \lor \neg \phi  && \text{by double negation} \\
\approx{} & \neg \psi \to \neg \phi && \text{by \tolor{}} \\
\end{align*}

  \item $\models (\phi \to \psi) \land (\psi \to \sigma) \to (\phi \to \sigma)$
%%% (b)
\begin{align*}
    & (\phi \colop{red}{\to} \psi) \land (\psi \colop{red}{\to} \sigma) \to (\phi \colop{red}{\to} \sigma) &&
\\ \approx{}
    & \neg ((\neg \phi \lor \psi) \land (\neg \psi \lor \sigma)) \lor \neg \phi \lor \sigma
    && \text{\textcolor{red}{by} $\to$/$\lor$,}
\\ \approx{}
    & ({\color{magenta}\phi \land \neg \psi}) \lor (\psi \land \neg \sigma) \lor \neg \phi \lor \sigma
    && \text{by De Morgan,}
\\ \approx{}
    & ({\color{magenta}\phantom{\neg}\phi} \lor (\psi \land \neg \sigma) \lor \neg \phi \lor \sigma) \colop{magenta}{\land} {} &&
\\  & ({\color{magenta}         \neg \psi} \lor (\psi \land \neg \sigma) \lor \neg \phi \lor \sigma)
    && \text{\textcolor{magenta}{by} distributivity,}
\\ \approx{}
    & (\phi \lor \neg \phi \lor \dots) \land {} &&
\\
    & (\neg \psi \lor ({\color{teal}\psi \land \neg \sigma}) \lor \neg \phi \lor \sigma)
    && \text{reordering the first conjunct by commutativity of $\lor$,}
\\ & && \text{now hiding most of that conjunct for clarity,}
\\ \approx{}
    & (\phi \lor \neg \phi \lor \dots) \land {} &&
\\  & (\neg \psi \lor {\color{teal}\phantom{\neg}\psi}   \lor \neg \phi \lor \sigma) \land {} &&
\\  & (\neg \psi \lor {\color{teal}         \neg \sigma} \lor \neg \phi \lor \sigma)
    && \text{\textcolor{teal}{by} distributivity,}
\\ \approx {}
    & (\bm{\phi \lor \neg \phi} \lor \dots) \land {} &&
\\  & (\bm{\psi \lor \neg \psi} \lor \neg \phi \lor \sigma) \land {} &&
\\  & (\bm{\sigma \lor \neg \sigma} \lor \neg \psi \lor \neg \phi)
    && \text{by commutativity of $\lor$,}
\\ \approx {}
    & \top \land \top \land \top \approx \top
    && \text{since all three conjuncts are now tautologies by excluded middle.}
\end{align*}

  \item $\models (\phi \to (\psi \land \neg \psi) \to \neg \phi$
%%% (c)
\begin{align*}
  & (\phi \to (\psi \land \neg \psi)) \to \neg \phi &&
\\ \approx{}
  & (\phi \to \bot) \to \neg \phi
  && \text{by Theorem 1.3.4 (g),}
\\ \approx{}
  & \neg \phi \to \neg \phi
  && \text{by Theorem 1.3.4 (f),}
\\ \approx{}
  & \top && \text{by reflexivity of implication.}
\end{align*}

  \item $\models (\phi \to \neg \psi) \to \neg \phi$
%%% (d)
\begin{align*}
  & (\phi \to \neg \phi) \to \neg \phi &&
\\ \approx{}
  & \neg (\neg \phi \vee \neg \phi) \vee \neg \phi
  && \text{by \tolor,}
\\ \approx{}
  & (\phi \wedge \phi) \vee \neg \phi
  && \text{by De Morgan,}
\\ \approx{}
  & \phi \lor \neg \phi
  && \text{by idempotence,}
\\ \approx{}
  & \top
  && \text{by excluded middle.}
\end{align*}

  \item $\models \neg (\phi \land \neg \phi)$
%%% (e)
\begin{align*}
  & \neg (\phi \land \neg \phi) &&
\\ \approx{}
  & (\neg \phi \lor \phi)
  && \text{by De Morgan,}
\\ \approx{}
  & \top
  && \text{by excluded middle.}
\end{align*}

  \item $\models \phi \to (\psi \to \phi \land \psi)$
%%% (f)
\begin{align*}
  & \phi \to (\psi \to \phi \land \psi) &&
\\ \approx{}
  & \neg \phi \lor \neg \psi \lor ({\color{red} \phi \land \psi})
  && \text{by \tolor,}
\\ \approx{}
    & (\neg \phi \lor \neg \psi \lor {\color{red}\phi}) \colop{red}{\land} {}
    &&
\\  & (\neg \phi \lor \neg \psi \lor {\color{red}\psi})
    && \text{\textcolor{red}{by} distributivity}
\\ \approx{}
    & (\bm{\neg \phi} \lor {\color{gray}\neg \psi} \lor \bm{\phi}) \land {}
    &&
\\  & ({\color{gray}\neg \phi} \lor \bm{\neg \psi \lor \psi})
    && \text{the two clauses are tautologies by excluded middle,}
\\ \approx{}
    & \top \wedge \top \approx \top.
    &&
\end{align*}

  \item $\models ((\phi \to \psi) \to \phi) \to \phi$
%%% (g)
\begin{align*}
  & ((\phi \to \psi) \to \phi) \to \phi &
\\ \approx{}
  & \neg (\neg (\neg \phi \lor \psi) \lor \phi) \lor \phi
  && \text{by \tolor,}
\\ \approx{}
  & ({\color{teal}(\neg \phi \lor \psi) \land \neg \phi}) \lor \phi
  && \text{by De Morgan,}
\\ \approx{}
  & ({\color{teal}(\neg \phi \lor \psi)} \lor \phi) \colop{teal}{\land} ({\color{teal}\neg \phi} \lor \phi)
  && \text{{\color{teal}by} distributivity,}
\\ \approx{}
  & (\neg \phi \lor {\color{gray}\psi} \lor \phi) \land (\neg \phi \lor \phi)
  &&
\\ \approx{}
  & \top \land \top \approx \top
  && \text{by excluded middle.}
\end{align*}
\end{enumerate}
\end{solution}

\begin{problem}[1.3.3] Show that $\{ \neg \}$ is not a functionally complete set of connectives. Same for $\{\to, \lor\}$ (hint: show that each formula $\phi$ with only $\to$ and $\lor$ there is a valuation v such that $\llbracket \phi \rrbracket_v = 1$)
\end{problem}
\begin{solution}
  For $\{ \neg \}$, that it is not functionally complete follows from the claim that propositions using only the connective $\neg$ contain exactly one atom. This can be proved by induction on the rank of the proposition. Then it is clear that 2-ary connectives that depend on the values of both inputs cannot be represented using only $\neg$.

  For $\{\to, \lor\}$, we prove by induction that every proposition $\phi$ using only connectives $\to$ and $\lor$, there is a valuation $v$ such that $\den{\phi} = 1$. From this it follows that we cannot define connectives always produce 0 regardless of input.

  In the case of an atom $p$, then the valuation $v$ where $v(p) = 1$ suffices. (Note: here the atoms can't include $\bot$ since if it did, we could define $\neg$ using $\neg p = p \to \bot$)

  In the case of $\phi \to \psi$, by the inductive hypothesis there is a valuation $v$ where $\den{\psi} = 1$. Then this valuation, extended to arbitrary values for atoms not in $\psi$, suffices for $\phi \to \psi$ as well.

  In the case of $\phi \lor \psi$, by the inductive hypothesis there is a valuation $v$ where $\den{\phi} = 1$. Then this valuation, extended to arbitrary values for atoms not in $\phi$, suffices for $\phi \lor \psi$ as well.
\end{solution}

\begin{problem}[1.3.4] Show that the Sheffer stroke, $\uparrow$, forms a functionally complete set. (hint: $\models \neg \phi \leftrightarrow \phi \uparrow \phi$).
\end{problem}
\begin{solution}
  In this solution, I will assert a series of equivalences for propositional formulae. All of them have them verified through checking their truth tables. In the interest of my time, your time, and my respect for your intelligence, I have not provided them.

  Note that the Sheffer stroke is equivalent to the logical operator familiar to many of us as NAND. Thus it should be unsurprising that $\phi \uparrow phi \approx \lnot phi$. From now on, I will assume that we have an $\lnot$ operator as we can simulate one through the previous equivalence.
  
  We can, of course, simulate $\land$ using $\lnot$ and the Sheffer stroke, equivalent to NAND. $\phi \land \psi \approx \lnot (\phi \uparrow \psi)$.

  Now we can simulate $\lor$ through the following equivalence. $\phi \lor \psi \approx \lnot (\lnot \phi \land \lnot \psi)$.

  Finally we can simulate $\to$ with the following equivalence. $\phi \to \psi \approx \lnot \phi \lor \psi$.

  A full unfolding of each of these simulations shows how each constructor of logical formulae can be simulated with the Sheffer stroke.

  With these simulations, we can prove that the Sheffer stroke forms a functionally complete set by induction on the structure of propositional logic formulae, or whatever the theorem that is equivalent to it is called in this chapter.

  For the case of atomic formulae, or proposition is trivially true. For all others involving argument formulae $\phi,\psi$, apply the inductive hypothesis to obtain $\phi'\approx \phi$ and $\psi' \approx \psi$ where $\phi',\psi'$ contain only the Sheffer stroke. Substitute these formulae in for the originals. Then use the corresponding simulation to rewrite the current expression without the use of the single non-Sheffer arrow operator left.
\end{solution}


\begin{problem}[1.3.10] Determine conjunctive and disjunctive normal forms for
\begin{enumerate}[(a)]
    \item $\neg (\phi \leftrightarrow \psi)$
    \item $((\phi \to \psi) \to \psi) \to \psi$
    \item $(\phi \to (\phi \land \neg \psi)) \land (\psi \to (\psi \land \neg \phi))$
\end{enumerate}
\end{problem}
\begin{solution}
For the first two formulae, I will provide truth tables and derive the CNF and DNF from the table.\\
\begin{center}
\begin{tabular}{|c|c|c|}
  \hline
  $\phi$ & $\psi$ & $\lnot (\phi \leftrightarrow \psi)$\\
  \hline
  0 & 0 & 0 \\
  \hline
  0 & 1 & 1 \\
  \hline
  1 & 0 & 1\\
  \hline
  1 & 1 & 0\\
  \hline
\end{tabular}
\end{center}  

From this table we can see that $\lnot (\phi \leftrightarrow \psi)$ is equivalent to $\lnot \phi \lor \psi$ which is in both disjunctive normal form, it is the disjunction of two singleton conjunctions, and conjective normal form, it is the singleton conjunction of the disjunction $\lnot \phi \lor \psi$.

\begin{center}
\begin{tabular}{|c|c|c|}
  \hline
  $\phi$ & $\psi$ & $((\phi \to \psi) \to \phi) \to \psi$\\
  \hline
  0 & 0 & 1 \\
  \hline
  0 & 1 & 1 \\
  \hline
  1 & 0 & 0\\
  \hline
  1 & 1 & 1\\
  \hline
\end{tabular}
\end{center}  

From this table, we can see that  $((\phi \to \psi) \to \phi) \to \psi$ is equivalent to $\lnot \phi \lor \psi$. This expression is also in both DNF and CNF.

\begin{center}
\begin{tabular}{|c|c|c|}
  \hline
  $\phi$ & $\psi$ &  $(\phi \to (\phi \land \neg \psi)) \land (\psi \to (\psi \land \neg \phi))$\\
  \hline
  0 & 0 & 1 \\
  \hline
  0 & 1 & 0 \\
  \hline
  1 & 0 & 1\\
  \hline
  1 & 1 & 1\\
  \hline
\end{tabular}
\end{center}  

This formula is equivalent to $\phi \lor \lnot \psi$. This expression is in both DNF and CNF.

\end{solution}

\begin{problem}[1.3.11] Give a criterion for a conjunctive normal form to be a tautology.
\end{problem}
\begin{solution}
\end{solution}

\begin{problem}[1.4.1] Show that the following propositions are derivable (note: see at the bottom for guidance on formatting proof trees)
\begin{enumerate}[(a)]
    \item $\varphi \to \varphi$
    \item $\bot \to \varphi$
    \item $\neg (\varphi \land \neg \varphi)$
    \item $(\varphi \to \psi) \leftrightarrow \neg (\varphi \land \neg \psi)$
    \item $(\varphi \land \psi) \leftrightarrow \neg (\varphi \to \neg \psi)$
    \item $\varphi \to (\psi \to (\varphi \land \psi))$
\end{enumerate}
\end{problem}
\begin{solution}\strut
\begin{enumerate}[(a)]

    \item\strut
    \begin{prooftree}
        \AxiomC{$\bracks{\varphi}^1$}

        \RightLabel{$\to I_1$}
        \UnaryInfC{$\varphi \to \varphi$}
    \end{prooftree}

    \item\strut
    \begin{prooftree}
        \AxiomC{$\bracks{\bot}^1$}

        \RightLabel{$\bot$}
        \UnaryInfC{$\varphi$}

        \RightLabel{$\to I_1$}
        \UnaryInfC{$\bot \to \varphi$}
    \end{prooftree}

    \item Here is (c) for notational reference:
    \begin{prooftree}
        \AxiomC{$[ \varphi \land \neg \varphi]^1$}
        \RightLabel{$\land E (l)$}
        \UnaryInfC{$\varphi$}
        \AxiomC{$[ \varphi \land \neg \varphi]^1$}
        \RightLabel{$\land E (r)$}
        \UnaryInfC{$\neg \varphi$}
        \RightLabel{$\to E$}
        \BinaryInfC{$\bot$}
        \RightLabel{$\to I_1$}
        \UnaryInfC{$\neg (\varphi \land \neg \varphi)$}
    \end{prooftree}

    \item Let
    {
        \AxiomC{$\mathcal D_1$}
        \noLine
        \UnaryInfC{$\parens{\varphi \to \psi} \to \lnot\parens{\varphi \land \lnot \psi}$}
        \DisplayProof
    }
    $=$
    \begin{prooftree}
                \AxiomC{$\bracks{\varphi \land \lnot \psi}^1$}
                \RightLabel{$\land E (l)$}
                \UnaryInfC{$\varphi$}

                \AxiomC{$\bracks{\varphi \to \psi}^2$}

            \RightLabel{$\to E$}
            \BinaryInfC{$\psi$}

            \AxiomC{$\bracks{\varphi \land \lnot \psi}^1$}
            \RightLabel{$\land E (r)$}
            \UnaryInfC{$\lnot \psi$}

        \RightLabel{$\to E$}
        \BinaryInfC{$\bot$}

        \RightLabel{$\to I_1$}
        \UnaryInfC{$\lnot\parens{\varphi \land \lnot \psi}$}

        \RightLabel{$\to I_2$}
        \UnaryInfC{$\parens{\varphi \to \psi} \to \lnot\parens{\varphi \land \lnot \psi}$}
    \end{prooftree}
    and let
    {
        \AxiomC{$\mathcal D_2$}
        \noLine
        \UnaryInfC{$\lnot \parens{\varphi \land \lnot \psi} \to \parens{\varphi \to \psi}$}
        \DisplayProof
    }
    $=$
    \begin{prooftree}
                \AxiomC{$\bracks{\varphi}^4$}
                \AxiomC{$\bracks{\lnot \psi}^3$}

            \RightLabel{$\land I$}
            \BinaryInfC{$\phi \land \lnot \psi$}

            \AxiomC{$\bracks{\lnot\parens{\varphi \land \lnot \psi}}^5$}

        \RightLabel{$\to E$}
        \BinaryInfC{$\bot$}

        \RightLabel{RAA${}_3$}
        \UnaryInfC{$\psi$}

        \RightLabel{$\to I_4$}
        \UnaryInfC{$\varphi \to \psi$}

        \RightLabel{$\to I_5$}
        \UnaryInfC{$\lnot \parens{\varphi \land \lnot \psi} \to \parens{\varphi \to \psi}$}
    \end{prooftree}
    Then
    \begin{prooftree}
            \AxiomC{$\mathcal D_1$}
            \noLine
            \UnaryInfC{$\parens{\varphi \to \psi} \to \lnot\parens{\varphi \land \lnot \psi}$}

            \AxiomC{$\mathcal D_2$}
            \noLine
            \UnaryInfC{$\lnot \parens{\varphi \land \lnot \psi} \to \parens{\varphi \to \psi}$}

        \RightLabel{$\land I$}
        \BinaryInfC{$\parens{\varphi \to \psi} \leftrightarrow \lnot\parens{\varphi \land \lnot \psi}$}
    \end{prooftree}

    \item Let
    {
        \AxiomC{$\mathcal D_1$}
        \noLine
        \UnaryInfC{$\parens{\varphi \land \psi} \to \lnot\parens{\varphi \to \lnot \psi}$}
        \DisplayProof
    }
    $=$
    \begin{prooftree}
            \AxiomC{$\bracks{\varphi \land \psi}^2$}
            \RightLabel{$\land E (r)$}
            \UnaryInfC{$\psi$}

                \AxiomC{$\bracks{\varphi \land \psi}^2$}
                \RightLabel{$\land E (l)$}
                \UnaryInfC{$\varphi$}

                \AxiomC{$\bracks{\varphi \to \lnot \psi}^1$}

            \RightLabel{$\to E$}
            \BinaryInfC{$\lnot \psi$}

        \RightLabel{$\to E$}
        \BinaryInfC{$\bot$}

        \RightLabel{$\to I_1$}
        \UnaryInfC{$\lnot\parens{\varphi \to \lnot \psi}$}

        \RightLabel{$\to I_2$}
        \UnaryInfC{$\parens{\varphi \land \psi} \to \lnot\parens{\varphi \to \lnot \psi}$}
    \end{prooftree}
    let
    {
        \AxiomC{$\lnot\parens{\varphi \to \lnot \psi}$}
        \noLine
        \UnaryInfC{$\mathcal D_2$}
        \noLine
        \UnaryInfC{$\varphi$}
        \DisplayProof
    }
    $=$
    \begin{prooftree}
                \AxiomC{$\bracks{\varphi}^3$}
                \AxiomC{$\bracks{\lnot \varphi}^4$}

            \RightLabel{$\to E$}
            \BinaryInfC{$\bot$}
            \RightLabel{$\bot$}
            \UnaryInfC{$\lnot \psi$}
            \RightLabel{$\to I_3$}
            \UnaryInfC{$\varphi \to \lnot \psi$}

            \AxiomC{$\lnot\parens{\varphi \to \lnot \psi}$}

        \RightLabel{$\to E$}
        \BinaryInfC{$\bot$}

        \RightLabel{RAA${}_4$}
        \UnaryInfC{$\varphi$}
    \end{prooftree}
    and let
    {
        \AxiomC{$\lnot\parens{\varphi \to \lnot \psi}$}
        \noLine
        \UnaryInfC{$\mathcal D_3$}
        \noLine
        \UnaryInfC{$\psi$}
        \DisplayProof
    }
    $=$
    \begin{prooftree}
            \AxiomC{$\bracks{\lnot \psi}^5$}
            \RightLabel{$\to I$}
            \UnaryInfC{$\varphi \to \lnot \psi$}

            \AxiomC{$\lnot\parens{\varphi \to \lnot \psi}$}

        \RightLabel{$\to E$}
        \BinaryInfC{$\bot$}

        \RightLabel{RAA${}_5$}
        \UnaryInfC{$\psi$}
    \end{prooftree}
    Then
    \begin{prooftree}
            \AxiomC{$\mathcal D_1$}
            \noLine
            \UnaryInfC{$\parens{\varphi \land \psi} \to \lnot\parens{\varphi \to \lnot \psi}$}

                \AxiomC{$\bracks{\lnot\parens{\varphi \to \lnot \psi}}^6$}
                \noLine
                \UnaryInfC{$\mathcal D_2$}
                \noLine
                \UnaryInfC{$\varphi$}

                \AxiomC{$\bracks{\lnot\parens{\varphi \to \lnot \psi}}^6$}
                \noLine
                \UnaryInfC{$\mathcal D_3$}
                \noLine
                \UnaryInfC{$\psi$}

            \RightLabel{$\land I$}
            \BinaryInfC{$\varphi \land \psi$}
            \RightLabel{$\to I_6$}
            \UnaryInfC{$\lnot \parens{\varphi \to \lnot \psi} \to \parens{\varphi \land \psi}$}

        \RightLabel{$\land I$}
        \BinaryInfC{$\parens{\varphi \land \psi} \leftrightarrow \lnot\parens{\varphi \to \lnot \psi}$}
    \end{prooftree}

    \item\strut
    \begin{prooftree}
            \AxiomC{$\bracks{\varphi}^2$}

            \AxiomC{$\bracks{\psi}^1$}

        \RightLabel{$\land I$}
        \BinaryInfC{$\varphi \land \psi$}

        \RightLabel{$\to I_1$}
        \UnaryInfC{$\psi \to \parens{\varphi \land \psi}$}

        \RightLabel{$\to I_2$}
        \UnaryInfC{$\varphi \to \parens{\psi \to \parens{\varphi \land \psi}}$}
    \end{prooftree}
\end{enumerate}
\end{solution}

\begin{problem}[1.4.3] Show
    \begin{enumerate}[(a)]
        \item $\varphi \vdash \neg (\neg \varphi \land \psi)$
        \item $\neg (\varphi \land \neg \psi), \varphi \vdash \psi$
        \item $\neg \varphi \vdash (\varphi \to \psi) \leftrightarrow \neg \varphi$
        \item $\vdash \varphi \implies \vdash \psi \to \varphi$
        \item $\neg \varphi \vdash \varphi \to \psi$
    \end{enumerate}
\end{problem}
\begin{solution}\strut
\begin{enumerate}[(a)]
    \item\strut
    \begin{prooftree}
            \AxiomC{$\varphi$}

                \AxiomC{$\bracks{\lnot \varphi \land \psi}^1$}
                \RightLabel{$\land E (l)$}

            \UnaryInfC{$\lnot \varphi$}

        \RightLabel{$\to E$}
        \BinaryInfC{$\bot$}

        \RightLabel{$\to I_1$}
        \UnaryInfC{$\lnot \parens{\lnot \varphi \land \psi}$}
    \end{prooftree}

    \item\strut
    \begin{prooftree}
                \AxiomC{$\varphi$}
                \AxiomC{$\bracks{\lnot \psi}^1$}

            \RightLabel{$\land I$}
            \BinaryInfC{$\varphi \land \lnot \psi$}

            \AxiomC{$\lnot \parens{\varphi \land \lnot \psi}$}

        \RightLabel{$\to E$}
        \BinaryInfC{$\bot$}

        \RightLabel{RAA${}_1$}
        \UnaryInfC{$\psi$}
    \end{prooftree}

    \item\strut
    \begin{prooftree}
            \AxiomC{$\lnot \varphi$}
            \RightLabel{$\to I$}
            \UnaryInfC{$\parens{\varphi \to \psi} \to \lnot \varphi$}

                \AxiomC{$\bracks{\varphi}^1$}
                \AxiomC{$\bracks{\lnot \varphi}^2$}

            \RightLabel{$\to E$}
            \BinaryInfC{$\bot$}
            \RightLabel{$\bot$}
            \UnaryInfC{$\psi$}
            \RightLabel{$\to I_1$}
            \UnaryInfC{$\varphi \to \psi$}
            \RightLabel{$\to I_2$}
            \UnaryInfC{$\lnot\varphi \to \parens{\varphi \to \psi}$}

        \RightLabel{$\land I$}
        \BinaryInfC{$\parens{\varphi \to \psi} \leftrightarrow \lnot \varphi$}
    \end{prooftree}

    \item
    Assume there is a derivation
    {\AxiomC{$\mathcal D$} \noLine \UnaryInfC{$\varphi$} \DisplayProof}.
    \begin{prooftree}
        \AxiomC{$\mathcal D$}

        \noLine
        \UnaryInfC{$\varphi$}

        \RightLabel{$\to I$}
        \UnaryInfC{$\psi \to \varphi$}
    \end{prooftree}

    \item\strut
    \begin{prooftree}
        \AxiomC{$\bracks{\varphi}^1$}
        \AxiomC{$\lnot \varphi$}

        \RightLabel{$\to E$}
        \BinaryInfC{$\bot$}

        \RightLabel{$\bot$}
        \UnaryInfC{$\psi$}

        \RightLabel{$\to I_1$}
        \UnaryInfC{$\varphi \to \psi$}
    \end{prooftree}
\end{enumerate}
\end{solution}

\begin{problem}[1.4.4] Show
    \begin{enumerate}[(a)]
        \item $[(\phi \to \psi) \to (\phi \to \sigma)] \to [ \phi \to \psi \to \sigma]$
        \item $((\phi \to \psi) \to \phi) \to \phi$
    \end{enumerate}
\end{problem}
\begin{solution}
\newcommand*\tagone{{\color{red}1}}
\newcommand*\tagtwo{{\color{teal}2}}
\newcommand*\tagthree{{\color{olive}3}}
% Colored "cancelling brackets"
\newcommand*\colorcancel[3]{{\colorlet{savedcolor}{.}\color{#1}\left\lbrack\color{savedcolor}{#3}\color{#1}\right\rbrack}^{\color{#1}#2}}
\newcommand*\cancelone[1]{\colorcancel{red}{1}{#1}}
\newcommand*\canceltwo[1]{\colorcancel{teal}{2}{#1}}
\newcommand*\cancelthree[1]{\colorcancel{olive}{3}{#1}}
%
  \begin{enumerate}[(a)]
    \item\strut
%%% (a)
\begin{prooftree}
          \AxiomC{$\canceltwo{\phi}$}
            \AxiomC{$\cancelthree{\psi}$}
            \RightLabel{${\to}I$}
            \UnaryInfC{$\phi \to \psi$}
            \AxiomC{$\cancelone{(\phi \to \psi) \to (\phi \to \sigma)}$}
          \RightLabel{${\to}E$}
          \BinaryInfC{$\phi\to\sigma$}
        \RightLabel{${\to}E$}
        \BinaryInfC{$\sigma$}
      \RightLabel{${\to}I_\tagthree{}$}
      \UnaryInfC{$\psi \to \sigma$}
    \RightLabel{${\to}I_\tagtwo{}$}
    \UnaryInfC{$\phi \to \psi \to \sigma$}
  \RightLabel{${\to}I_\tagone{}$}
  \UnaryInfC{$[(\phi \to \psi) \to (\phi \to \sigma)] \to [ \phi \to \psi \to \sigma]$}
\end{prooftree}

    \item\strut
%%% (b)
\begin{prooftree}
              \AxiomC{$\cancelthree{\phi}$}
              \AxiomC{$\canceltwo{\neg\phi}$}
              \RightLabel{${\to} E$}
              \BinaryInfC{$\bot$}
            \RightLabel{$\bot$}
            \UnaryInfC{$\psi$}
          \RightLabel{${\to}I_\tagthree{}$}
          \UnaryInfC{$\phi \to \psi$}
          \AxiomC{$\cancelone{(\phi \to \psi) \to \phi}$}
        \RightLabel{${\to}E$}
        \BinaryInfC{$\phi$}
        \AxiomC{$\canceltwo{\neg\phi}$}
      \RightLabel{${\to}E$}
      \BinaryInfC{$\bot$}
    \RightLabel{$\text{RAA}_\tagtwo{}$}
    \UnaryInfC{$\phi$}
  \RightLabel{${\to} I_\tagone{}$}
  \UnaryInfC{$((\phi \to \psi) \to \phi) \to \phi$}
\end{prooftree}
  \end{enumerate}
\end{solution}

\begin{problem}[1.4.10] Give an inductive definition of the relation $\vdash$ (c.f. Lemma 1.4.3). Show this relation coincides with the derived relation of Definition 1.4.2. Conclude that if $\Gamma \vdash \phi$, then $\Gamma$ contains a finite $\Delta$ such that $\Delta \vdash \phi$.
\end{problem}
\begin{solution}
\end{solution}


\begin{problem}[1.5.1] Check which of the following sets are consistent
    \begin{enumerate}[(a)]
        \item $\{\neg p_1 \land p_2 \to p_0, p_1 \to (\neg p_1 \to p_2), p_0 \leftrightarrow \neg p_2\}$
        \item $\{p_0 \to p_1. p_1 \to p_2, p2 \to p_3, p_3 \to \neg p_0\}$
        \item $\{p_0 \to p_1, p_0 \land p_2 \to p_1 \land p_3, p_0 \land p_2 \land p_4 \to p_1 \land p_3 \land p_5 \}$
    \end{enumerate}
\end{problem}
\begin{solution}
\end{solution}

\begin{problem}[1.5.2] Show that the following are equivalent:
    \begin{enumerate}[(a)]
        \item $\{\phi_1 \ldots \phi_n\}$ is consistent
        \item $\not \vdash \neg (\phi_1 \land \ldots \land \phi_n)$
        \item $\not \vdash \phi_1 \land \phi_2 \ldots \land \phi_{n-1}\to \neg \phi_n$
    \end{enumerate}
\end{problem}
\begin{solution}
\end{solution}

\begin{problem}[1.5.3] $\phi$ is \textit{independent} from $\Gamma$ if
    $\Gamma \not \vdash \phi$ and $\Gamma \not \vdash \neg \phi$. Show that $p_1 \to p_2$ is independent from $\{p_1 \leftrightarrow p_0 \land \neg p_2, p_2 \to p_0\}$
\end{problem}
\begin{solution}
\end{solution}


\begin{problem}[1.5.6] Show that a consistent set $\Gamma$ is maximally consistent if either 
  $\phi \in \Gamma$ or $\neg \phi \in \Gamma$ for all $\phi$.
\end{problem}
\begin{solution}
We follow with proof by contradiction. Assume that $\Gamma$ is not maximally consistent given these
conditions. Then, there must exist some $\Gamma ' $ such that $\Gamma \subseteq \Gamma '$, where
$\Gamma '$ is consistent and $\Gamma \neq \Gamma '$. This implies that there must exist some
$\phi$ such that $\phi \in \Gamma'$ and $\phi \notin \Gamma$. Since we assume that for all $\phi$,
either $\phi \in \Gamma$ or $\neg \phi \in \Gamma$ must hold, $\neg \phi \in \Gamma$.
Since $\Gamma$ is a subset of $\Gamma'$, all elements of $\Gamma$ must be included in $\Gamma'$.
Thus, ${\phi, \neg \phi}\in \Gamma'$, which is a contradiction since we assumed $\Gamma'$ to be
consistent. This shows that $\Gamma$ is a maximally consistent set.
\end{solution}


\begin{problem}[1.5.10] Show $\textrm{Cons}(\Gamma) = \{ \sigma | \Gamma \vdash \sigma \}$ is maximally consistent if and only if $\Gamma$ is complete.
\end{problem}
\begin{solution}

  Problem 1.5.6 shows that $\textrm{Cons}(\Gamma) = \{ \sigma | \Gamma \vdash \sigma \}$ is maximally
  consistent if $\Gamma$ is a complete set.
  
  Now, let's show that $\Gamma$ is a complete set if $\textrm{Cons}(\Gamma) = \{ \sigma | \Gamma 
  \vdash \sigma \}$ is maximally consistent. Since $\textrm{Cons}(\Gamma)$ is maximally consistent, 
  by Lemma 1.5.8., $\textrm{Cons}(\Gamma)$ is closed under derivability, so $Cons(\Gamma) \vdash \phi \Rightarrow $
  $\phi \in Cons(\Gamma)$. Since this set is consistent, there does not exist a $\phi$ such that
  $\Gamma \vdash \phi \land \Gamma \vdash \neg \phi$. Thus, forall $\phi$, either $\phi \in \Gamma$
  or $\neg \phi \in \Gamma$.
\end{solution}


\section{Notation Reference}

\begin{table}[H]
    \begin{tabular}{ll}
        \textbf{Notation} & \textbf{Usual meaning} \\
        $\to$ $\leftarrow$ $\lor$ $\land$ $\leftrightarrow$ $\neg$ & Object-level connectives \\
        $\uparrow$ & Scheffer stroke (nand)\\
        $\top$, $\bot$ & Object-level true, false \\
        $\implies$ $\impliedby$ $\iff$ $\neg$ & Meta-level connectives \\

        $\llbracket \phi \rrbracket_v$ & denotation of $\phi$ under valuation $v$\\
        $\phi \vdash \psi$ & $\phi$ proves $\psi$ \\
        $\phi \models \psi$ & $\phi$ models $\psi$ \\
        $\ulcorner \phi \urcorner$ & Quine quotations/G\"odel numbering \\
        $\bigwedge, \bigvee$ & \\
        $\bigdoublevee$, $\bigdoublewedge$ & See Section 1.3 \\
        $\cup$, $\cap$, $\subset$, $\subseteq$, $\subsetneq$ & Set operations \\
        $\sqcup$, $\sqcap$, $\sqsubset$, $\sqsubseteq$ & Misc \\
\end{tabular}
\end{table}

\subsection{Typesetting proof trees}

\begin{prooftree}
    \AxiomC{}
    \RightLabel{identity}
    \UnaryInfC{$X \to X$}
\end{prooftree}

\begin{prooftree}
\AxiomC{$H$}
\RightLabel{\scriptsize{$\lor$ introduction left}}
\UnaryInfC{$H \lor G$}
\end{prooftree}

\begin{prooftree}
    \AxiomC{$H$}
    \AxiomC{$G$}
    \RightLabel{\scriptsize{$\land$ introduction}}
    \BinaryInfC{$H \land G$}
\end{prooftree}


\begin{prooftree}
    \AxiomC{$A$}
    \AxiomC{$B$}
    \AxiomC{$C$}
    \RightLabel{\scriptsize{weird rule}}
    \TrinaryInfC{$D$}
\end{prooftree}

\end{document}
